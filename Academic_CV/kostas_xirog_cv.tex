%%%%%%%%%%%%%%%%%%%%%%%%%%%%%%%%%%%%%%%%%
% Medium Length Graduate Curriculum Vitae
% LaTeX Template
% Version 1.1 (9/12/12)
%
% This template has been downloaded from:
% http://www.LaTeXTemplates.com
%
% Original author:
% Rensselaer Polytechnic Institute (http://www.rpi.edu/dept/arc/training/latex/resumes/)
%
% Important note:
% This template requires the res.cls file to be in the same directory as the
% .tex file. The res.cls file provides the resume style used for structuring the
% document.
%
%%%%%%%%%%%%%%%%%%%%%%%%%%%%%%%%%%%%%%%%%

%----------------------------------------------------------------------------------------
%	PACKAGES AND OTHER DOCUMENT CONFIGURATIONS
%----------------------------------------------------------------------------------------

\documentclass[margin, 10pt]{res} % Use the res.cls style, the font size can be changed to 11pt or 12pt here

\usepackage{helvet} % Default font is the helvetica postscript font
%\usepackage{newcent} % To change the default font to the new century schoolbook postscript font uncomment this line and comment the one above
\usepackage{hyperref}
\hypersetup{
     colorlinks   = true
     }

\setlength{\textwidth}{5.1in} % Text width of the document

\begin{document}

%----------------------------------------------------------------------------------------
%	NAME AND ADDRESS SECTION
%----------------------------------------------------------------------------------------

\moveleft.5\hoffset\centerline{\large\bf Konstantinos Xirogiannopoulos} % Your name at the top

\moveleft\hoffset\vbox{\hrule width\resumewidth height 1pt}\smallskip % Horizontal line after name; adjust line thickness by changing the '1pt'

\moveleft.5\hoffset\centerline{3226 A.V. Williams Bldg} % Your address
\moveleft.5\hoffset\centerline{College Park, MD 20740}
\moveleft.5\hoffset\centerline{kostasx@cs.umd.edu}

%----------------------------------------------------------------------------------------

\begin{resume}

%----------------------------------------------------------------------------------------
%	OBJECTIVE SECTION
%----------------------------------------------------------------------------------------

\section{INTERESTS}

I am interested in database systems, big data management, large-scale data analytics and distributed systems

%----------------------------------------------------------------------------------------
%	EDUCATION SECTION
%----------------------------------------------------------------------------------------

\section{EDUCATION}

{\sl \textbf{PhD. in Computer Science},}  \hfill \textbf{August 2014 - currently} \\
University of Maryland, College Park\\
\textit{Advisor:} Prof. Amol Deshpande

{\sl \textbf{BSc. in Computer Science},}  \hfill \textbf{Sept. 2009 - Jan. 2014} \\
Athens University of Economics and Business\\
\textit{Ranked in the top 7.5\% in past five years of graduates}


%----------------------------------------------------------------------------------------
%	Technology SKILLS SECTION
%----------------------------------------------------------------------------------------

\section{HONORS \& AWARDS}

\begin{itemize}
  \item Dean's Fellowship \\
  \textit{University of Maryland, College Park}
  \item Honorary Scholarship and Award for Academic and Moral distinction on first year (2009-10)
 \textit{State Scholarships Foundation}\\
 (Ranked \#1 / 240)
\end{itemize}

%----------------------------------------------------------------------------------------
%	RESEARCH EXPERIENCE SECTION
%----------------------------------------------------------------------------------------

\section{RESEARCH EXPERIENCE}

{\sl \textbf{Graduate Research Assistant}} \hfill \textbf{Summer 2015} \\
University of Maryland, College Park \href{http://www.cs.umd.edu/~amol/DBGroup/pubs.html}{Databases Lab}\\
\textit{Advisor}: Prof. Amol Deshpande\\
Full-time Graduate Research Assistantship working on the followup research paper
for the \textit{GraphGen} Project.\\
\\
{\sl Large-Scale Graph Extraction from Relational Datasets}\\
\begin{itemize}
  \item GraphGen: Efficiently and intuitively extracting graphs from relational data using a custom Domain Specific Language based on Datalog. This allows users to conduct in-memory large-scale graph analytics on their relational datasets without the need for migrating to a native graph database
\end{itemize}

{\sl \textbf{Undergraduate Researcher}} \hfill \textbf{Oct. 2012 - Sept. 2013}\\
Athens University of Economics and Business\\
\textit{Thesis Title}:
\href{https://drive.google.com/open?id=0B20MIwp_I7FlVFlNVWtQb3VXTmM}{``Graph Databases and Big Data : Study, Overview of Existing Systems, and Sub-Graph Matching Queries Algorithm Implementation using Apache Hama Graph-Parallel Processing Framework"}\\
\textit{Supervisor}: Prof. Yannis Kotidis

%----------------------------------------------------------------------------------------
%	PUBLICATIONS SECTION
%----------------------------------------------------------------------------------------

\section{PUBLICATIONS }

{\sl \textbf{Demonstrations} }\\
\begin{itemize}
  \item Konstantinos Xirogiannopoulos, Udayan Khurana, Amol Deshpande\\
  \href{http://www.vldb.org/pvldb/vol8/p2032-xirogiannopoulos.pdf}{GraphGen: Exploring Interesting Graphs in Relational Data}\\
  \textit{VLDB 2015}
\end{itemize}

%----------------------------------------------------------------------------------------
%	INDUSTRY EXPERIENCE SECTION
%----------------------------------------------------------------------------------------

\section{INDUSTRIAL EXPERIENCE}

{\sl \textbf{Back-End Sofware Engineer Intern} } \hfill \textbf{March 2014 - June 2014}\\
Asuum GmbH based in Berlin, Germany\\
Developed for asynchronous inter-server communication, data management, data analysis and web application back-end maintenance.\\
\textit{Technical Skills}: Java, Hadoop, Map-Reduce, MongoDB, ElasticSearch, Jenkins

%----------------------------------------------------------------------------------------
%	TEACHING EXPERIENCE SECTION
%----------------------------------------------------------------------------------------

\section{TEACHING \& MENTORING}

{\sl \textbf{Graduate Teaching Assistant (Teaching)}} \hfill \textbf{Sept. 2014 - Dec. 2014}\\
\textit{Course}: CMSC132: Object Oriented Programming II \\
\textit{Instructor}: Larry Herman\\
Conducted four 1-hour long discussion sections per week (approx. 35 students per section) and held 4-hours of office-hours every week for answering questions. These discussion sections included explaining concepts, doing worksheets, assigning and grading quizzes. Graded quizzes and actively participated in grading of midterms and final exams.



%----------------------------------------------------------------------------------------
%	COURSES SECTION
%----------------------------------------------------------------------------------------

\section{RECENT COURSES}

{\sl \textbf{CMSC818:Distributed and Cloud Based File Systems}} \hfill \textbf{Sept. 2014 - Dec. 2014}\\
\textit{Instructor}: Prof. Peter Keleher\\
\\
\textit{Built Distributed Fault Tolerant, Durable File System from Scratch}: Starting from a simple in-memory file system implementation, made it persistent, and later applied and implemented distributed file systems concepts like versioning, replication and distributed consensus (Raft) and developed a fully functional, distributed file system with many guarantees. All development done in \textit{Go} (``A").

{\sl \textbf{CMSC723:Computational Linguistics I}} \hfill \textbf{Sept. 2014 - Dec. 2014}\\
\textit{Instructor}: Prof. Hal Daume III\\
\begin{itemize}
  \item
  Konstantinos Xirogiannopoulos, Kasia Hitczenko\\
  \textit{Automatic Quiz-bowl Question Answering}: Built a classifier that classified text quiz-bowl questions to their correct answers in an automatic way. Text processing done in \textit{Python} (``A+").
\end{itemize}


{\sl \textbf{CMSC 724: Database Management Systems}} \hfill \textbf{Feb. 2015 - May. 2015}\\
\textit{Instructor}: Prof. Amol Deshpande\\
\begin{itemize}
  \item Konstantinos Xirogiannopoulos, Benjamin Bengfort\\
  \href{https://drive.google.com/open?id=0B20MIwp_I7FlUGhyVmFYcjFuYmM}{Graph-Based Machine Learning on Relational Data} (``A+").
\end{itemize}

{\sl \textbf{CMSC 734: Information Visualization}} \hfill \textbf{Feb. 2015 - May. 2015}\\
\textit{Instructor}: Prof. Ben Schneiderman\\
\begin{itemize}
  \item Konstantinos Xirogiannopoulos, Myco Paulo, Zheng Xu, Deok Gun Park\\
  \href{https://wiki.cs.umd.edu/cmsc734_s15/images/f/fc/TimeGrouper_FinalReport.pdf}{TimeGrouper: Visualizing Time Series Clustering Towards the Identification of the Contributing Factors to the Global Decay Rate of Vulnerabilities} (``A")
\end{itemize}


%----------------------------------------------------------------------------------------
%	EXTRA-CURRICULAR ACTIVITIES SECTION
%----------------------------------------------------------------------------------------

\section{HOBBIES \& INTERESTS}

\textit{Music} (Electric \& Acoustic Guitar), \textit{Sports} (Basketball [participation in town tournaments] , Swimming, Skiing), \textit{Leisure} (Fishing, Cinema, Android OS Enthusiast, Rubik\’s Speedcubing [national contest participation])

%----------------------------------------------------------------------------------------

\end{resume}
\end{document}
