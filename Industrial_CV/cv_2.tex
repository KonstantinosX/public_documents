%%%%%%%%%%%%%%%%%%%%%%%%%%%%%%%%%%%%%%%%%
% Medium Length Graduate Curriculum Vitae
% LaTeX Template
% Version 1.1 (9/12/12)
%
% This template has been downloaded from:
% http://www.LaTeXTemplates.com
%
% Original author:
% Rensselaer Polytechnic Institute (http://www.rpi.edu/dept/arc/training/latex/resumes/)
%
% Important note:
% This template requires the res.cls file to be in the same directory as the
% .tex file. The res.cls file provides the resume style used for structuring the
% document.
%
%%%%%%%%%%%%%%%%%%%%%%%%%%%%%%%%%%%%%%%%%

%----------------------------------------------------------------------------------------
%	PACKAGES AND OTHER DOCUMENT CONFIGURATIONS
%----------------------------------------------------------------------------------------

\documentclass[margin, 10pt]{res} % Use the res.cls style, the font size can be changed to 11pt or 12pt here

\usepackage{helvet} % Default font is the helvetica postscript font
%\usepackage{newcent} % To change the default font to the new century schoolbook postscript font uncomment this line and comment the one above
% \usepackage{hyperref}
\usepackage{color}

\usepackage{hyperref}
\hypersetup{
  colorlinks,
  citecolor=Violet,
  linkcolor=Red}

\setlength{\textwidth}{5.1in} % Text width of the document

\begin{document}

%----------------------------------------------------------------------------------------
%	NAME AND ADDRESS SECTION
%----------------------------------------------------------------------------------------

\moveleft.5\hoffset\centerline{\large\bf Konstantinos Xirogiannopoulos} % Your name at the top

\moveleft\hoffset\vbox{\hrule width\resumewidth height 1pt}\smallskip % Horizontal line after name; adjust line thickness by changing the '1pt'

\moveleft.5\hoffset\centerline{3226 A.V. Williams Bldg} % Your address
\moveleft.5\hoffset\centerline{College Park, MD 20740}
\moveleft.5\hoffset\centerline{\href{mailto:kostasx@cs.umd.edu}{kostasx@cs.umd.edu}}

%----------------------------------------------------------------------------------------

\begin{resume}

%----------------------------------------------------------------------------------------
%	OBJECTIVE SECTION
%----------------------------------------------------------------------------------------

\section{INTERESTS}

I am interested in database systems, data management, large-scale analytics and distributed systems.

%----------------------------------------------------------------------------------------
%	EDUCATION SECTION
%----------------------------------------------------------------------------------------

\section{EDUCATION}

{\sl \textbf{Ph.D. in Computer Science},}  \hfill \textbf{August 2014 -- present} \\
University of Maryland, College Park,  \textit{MD, USA}\\

{\sl \textbf{B.Sc. in Computer Science},}  \hfill \textbf{September 2009 -- January 2014} \\
Athens University of Economics and Business, \textit{Athens Greece}\\
\textit{Ranked in the top 7.5\% of graduates in past five years}\\
% {\sl PhD. in Computer Science,}  \hfill August 2014 - August 2015 \\
% Athens University of Economics and Business, Athens, Greece\\

%----------------------------------------------------------------------------------------
%	Technology SKILLS SECTION
%----------------------------------------------------------------------------------------

\section{TECHNICAL \\ SKILLS}

{\sl Programming Languages:} Java, Go, Python, JavaScript, C. \\
{\sl Operating Systems:} Mac OS X, Windows, Ubuntu.\\
{\sl Back-end Technologies:} PostgreSQL, MySQL, MongoDB, ElasticSearch. \\
{\sl Web Front-end Technologies:} Bootstrap, JQuery, D3. \\
{\sl Data Analysis Technologies:} Hadoop, Hama, Giraph, Spark.\\
{\sl Software:} Git, Scikit-Learn, Heroku, Jenkins. \\




%----------------------------------------------------------------------------------------
%	RESEARCH EXPERIENCE SECTION
%----------------------------------------------------------------------------------------

\section{RESEARCH EXPERIENCE}

{\sl \textbf{Graduate Research Assistant}} \hfill \textbf{June 2015 -- present} \\
University of Maryland, College Park \href{http://www.cs.umd.edu/~amol/DBGroup/pubs.html}{Databases Lab}\\
\textit{Advisor}: Prof. Amol Deshpande\\
{\sl Graph Extraction from Relational Datasets}\\
  Efficiently and intuitively extracting graphs from relational data using a custom Domain Specific Language based on Datalog. Allows users to conduct in-memory large-scale graph analytics on their relational datasets without the need for complex ETL or migrating to a native graph database.\href{http://konstantinosx.github.io/graphgen-project/}{\textit{ [Project Webpage]}}

{\sl \textbf{Undergraduate Researcher}} \hfill \textbf{Oct. 2012 -- Sept. 2013}\\
Athens University of Economics and Business\\
\textit{Thesis Title}:
\href{https://drive.google.com/open?id=0B20MIwp_I7FlVFlNVWtQb3VXTmM}{``Graph Databases and Big Data : Study, Overview of Existing Systems, and Sub-Graph Pattern Matching Implementation using Apache Hama"}\\
\textit{Supervisor}: Prof. Yannis Kotidis

%----------------------------------------------------------------------------------------
%	PUBLICATIONS SECTION
%----------------------------------------------------------------------------------------

\section{PUBLICATIONS }

\begin{itemize}
   \item Konstantinos Xirogiannopoulos, Udayan Khurana, Amol Deshpande\\
  \href{http://www.vldb.org/pvldb/vol8/p2032-xirogiannopoulos.pdf}{GraphGen: Exploring Interesting Graphs in Relational Data} \textit{[\href{https://www.youtube.com/watch?v=GDVBLv-oedQ}{video}]}\\
  \textit{VLDB 2015, Demonstrations Track}
  \item Konstantinos Xirogiannopoulos, Amol Deshpande\\
  Efficient Graph Analytics over Relational Datasets\\
  \textit{ SIGMOD 2016, Research Paper Track} \textit{ [Pending Submission]}
\end{itemize}

%----------------------------------------------------------------------------------------
%	HONORS AND AWARDS SECTION
%----------------------------------------------------------------------------------------

\section{HONORS \& AWARDS}

% \begin{itemize}
   Dean's Fellowship \\
  \textit{University of Maryland, College Park}\\
  \\
   Honorary Scholarship and Award for Academic and Moral Distinction in first year at AUEB (2009-2010)
 \textit{State Scholarships Foundation IKY} {\sl \textbf{(Ranked \#1 / 240)}}
% \end{itemize}

%----------------------------------------------------------------------------------------
%	PROFESSIONAL EXPERIENCE SECTION
%----------------------------------------------------------------------------------------

\section{PROFESSIONAL EXPERIENCE}

{\sl \textbf{Back-End Sofware Engineer Intern} } \hfill \textbf{March 2014 -- June 2014}\\
\textit{\href{http://www.asuum.com/products/}{Asuum GmbH} based in Berlin, Germany}\\
Developed for asynchronous inter-server communication, data management, data analysis and web application back-end maintenance. Used \textit{minhashing} techniques to deal with duplicate news articles from publishers on the \textit{Asuum Bounce} project.\\
\textit{Skills}: \textbf{\textit{Java, Hadoop (MapReduce), MongoDB, ElasticSearch, Jenkins}}

%----------------------------------------------------------------------------------------
%	TEACHING EXPERIENCE SECTION
%----------------------------------------------------------------------------------------

\section{TEACHING \& MENTORING}

{\sl \textbf{Graduate Teaching Assistant}} \hfill \textbf{Sept. 2014 -- Dec. 2014}\\
\textit{Course}: CMSC132: Object Oriented Programming II \\
\textit{Instructor}: Mr. Larry Herman\\
Object Oriented Programming concepts, best practices, unit testing, basic data structures, graph algorithms, multi-threading and concurrency concepts (all taught in \textbf{Java}). Conducted four 1-hour long discussion sections/week (approx. 35 students/section) and held 4-hours of office-hours every week for answering questions. Graded quizzes, midterms and final exams, in collaboration with other Teaching Assistants.


%----------------------------------------------------------------------------------------
%	SELECTED SECTION
%----------------------------------------------------------------------------------------

\section{SELECTED COURSES \& PROJECTS}

{\sl \textbf{Distributed \& Cloud Based File Systems}} \hfill \textbf{Sept. 2014 -- Dec. 2014}\\
\textit{Instructor}: Prof. Peter Keleher
\begin{itemize}
  \item \textit{Built Distributed, Fault Tolerant, Durable File System from Scratch}: Starting from a simple in-memory file system implementation, made it persistent to disk, and implemented distributed file systems concepts like file versioning, replication and distributed consensus (Raft) towards a fully functional, distributed file system with the above guarantees. All development done in \textbf{\textit{Go}} (A).
\end{itemize}


{\sl \textbf{Computational Linguistics I}} \hfill \textbf{Sept. 2014 -- Dec. 2014}\\
\textit{Instructor}: Prof. Hal Daume III
\begin{itemize}
  \item
  Konstantinos Xirogiannopoulos, Kasia Hitczenko\\
  \textit{Built a multi-class classifier}: Classifying quizbowl question text to their correct answers in an automatic way using various training sets.\\ \textit{Skills}: \textbf{\textit{Python, Scikit-Learn, Nltk}} (A+).
\end{itemize}

{\sl \textbf{Database Management Systems}} \hfill \textbf{Feb. 2015 -- May. 2015}\\
\textit{Instructor}: Prof. Amol Deshpande
\begin{itemize}
  \item Konstantinos Xirogiannopoulos, Benjamin Bengfort\\
  \href{https://drive.google.com/open?id=0B20MIwp_I7FlUGhyVmFYcjFuYmM}{Graph-Based Machine Learning on Relational Data} (A+).
\end{itemize}

{\sl \textbf{Information Visualization}} \hfill \textbf{Feb. 2015 -- May. 2015}\\
\textit{Instructor}: Prof. Ben Schneiderman
\begin{itemize}
  \item Konstantinos Xirogiannopoulos, Myco Paulo, Zheng Xu, Deok Gun Park\\
  \href{https://wiki.cs.umd.edu/cmsc734_s15/images/f/fc/TimeGrouper_FinalReport.pdf}{TimeGrouper: Visualizing Time Series Clustering Towards Studying the Global Decay Rate of Vulnerabilities}  \textit{[\href{https://www.youtube.com/watch?v=oDgl6pp1CVU}{video demo}]}\\
  \textit{Implemented and deployed RESTful service} (Python/Flask) that applied clustering techniques to time series data on the fly, formatted and returned results to front-end application.  \textit{Skills}: \textbf{\textit{Python, Scikit-Learn, Flask, Heroku}} (A)
\end{itemize}


%----------------------------------------------------------------------------------------
%	EXTRA-CURRICULAR ACTIVITIES SECTION
%----------------------------------------------------------------------------------------

\section{HOBBIES \& INTERESTS}

\textbf{Music}: Electric \& Acoustic Guitar, \textbf{Sports}: Basketball, Swimming, Skiing, \textbf{Leisure}: Fishing, Cinema, Rubiks Speedcubing

%----------------------------------------------------------------------------------------

\end{resume}
\end{document}
